% Options for packages loaded elsewhere
\PassOptionsToPackage{unicode}{hyperref}
\PassOptionsToPackage{hyphens}{url}
%
\documentclass[
]{article}
\usepackage{amsmath,amssymb}
\usepackage{lmodern}
\usepackage{iftex}
\ifPDFTeX
  \usepackage[T1]{fontenc}
  \usepackage[utf8]{inputenc}
  \usepackage{textcomp} % provide euro and other symbols
\else % if luatex or xetex
  \usepackage{unicode-math}
  \defaultfontfeatures{Scale=MatchLowercase}
  \defaultfontfeatures[\rmfamily]{Ligatures=TeX,Scale=1}
\fi
% Use upquote if available, for straight quotes in verbatim environments
\IfFileExists{upquote.sty}{\usepackage{upquote}}{}
\IfFileExists{microtype.sty}{% use microtype if available
  \usepackage[]{microtype}
  \UseMicrotypeSet[protrusion]{basicmath} % disable protrusion for tt fonts
}{}
\makeatletter
\@ifundefined{KOMAClassName}{% if non-KOMA class
  \IfFileExists{parskip.sty}{%
    \usepackage{parskip}
  }{% else
    \setlength{\parindent}{0pt}
    \setlength{\parskip}{6pt plus 2pt minus 1pt}}
}{% if KOMA class
  \KOMAoptions{parskip=half}}
\makeatother
\usepackage{xcolor}
\usepackage[margin=1in]{geometry}
\usepackage{color}
\usepackage{fancyvrb}
\newcommand{\VerbBar}{|}
\newcommand{\VERB}{\Verb[commandchars=\\\{\}]}
\DefineVerbatimEnvironment{Highlighting}{Verbatim}{commandchars=\\\{\}}
% Add ',fontsize=\small' for more characters per line
\usepackage{framed}
\definecolor{shadecolor}{RGB}{248,248,248}
\newenvironment{Shaded}{\begin{snugshade}}{\end{snugshade}}
\newcommand{\AlertTok}[1]{\textcolor[rgb]{0.94,0.16,0.16}{#1}}
\newcommand{\AnnotationTok}[1]{\textcolor[rgb]{0.56,0.35,0.01}{\textbf{\textit{#1}}}}
\newcommand{\AttributeTok}[1]{\textcolor[rgb]{0.77,0.63,0.00}{#1}}
\newcommand{\BaseNTok}[1]{\textcolor[rgb]{0.00,0.00,0.81}{#1}}
\newcommand{\BuiltInTok}[1]{#1}
\newcommand{\CharTok}[1]{\textcolor[rgb]{0.31,0.60,0.02}{#1}}
\newcommand{\CommentTok}[1]{\textcolor[rgb]{0.56,0.35,0.01}{\textit{#1}}}
\newcommand{\CommentVarTok}[1]{\textcolor[rgb]{0.56,0.35,0.01}{\textbf{\textit{#1}}}}
\newcommand{\ConstantTok}[1]{\textcolor[rgb]{0.00,0.00,0.00}{#1}}
\newcommand{\ControlFlowTok}[1]{\textcolor[rgb]{0.13,0.29,0.53}{\textbf{#1}}}
\newcommand{\DataTypeTok}[1]{\textcolor[rgb]{0.13,0.29,0.53}{#1}}
\newcommand{\DecValTok}[1]{\textcolor[rgb]{0.00,0.00,0.81}{#1}}
\newcommand{\DocumentationTok}[1]{\textcolor[rgb]{0.56,0.35,0.01}{\textbf{\textit{#1}}}}
\newcommand{\ErrorTok}[1]{\textcolor[rgb]{0.64,0.00,0.00}{\textbf{#1}}}
\newcommand{\ExtensionTok}[1]{#1}
\newcommand{\FloatTok}[1]{\textcolor[rgb]{0.00,0.00,0.81}{#1}}
\newcommand{\FunctionTok}[1]{\textcolor[rgb]{0.00,0.00,0.00}{#1}}
\newcommand{\ImportTok}[1]{#1}
\newcommand{\InformationTok}[1]{\textcolor[rgb]{0.56,0.35,0.01}{\textbf{\textit{#1}}}}
\newcommand{\KeywordTok}[1]{\textcolor[rgb]{0.13,0.29,0.53}{\textbf{#1}}}
\newcommand{\NormalTok}[1]{#1}
\newcommand{\OperatorTok}[1]{\textcolor[rgb]{0.81,0.36,0.00}{\textbf{#1}}}
\newcommand{\OtherTok}[1]{\textcolor[rgb]{0.56,0.35,0.01}{#1}}
\newcommand{\PreprocessorTok}[1]{\textcolor[rgb]{0.56,0.35,0.01}{\textit{#1}}}
\newcommand{\RegionMarkerTok}[1]{#1}
\newcommand{\SpecialCharTok}[1]{\textcolor[rgb]{0.00,0.00,0.00}{#1}}
\newcommand{\SpecialStringTok}[1]{\textcolor[rgb]{0.31,0.60,0.02}{#1}}
\newcommand{\StringTok}[1]{\textcolor[rgb]{0.31,0.60,0.02}{#1}}
\newcommand{\VariableTok}[1]{\textcolor[rgb]{0.00,0.00,0.00}{#1}}
\newcommand{\VerbatimStringTok}[1]{\textcolor[rgb]{0.31,0.60,0.02}{#1}}
\newcommand{\WarningTok}[1]{\textcolor[rgb]{0.56,0.35,0.01}{\textbf{\textit{#1}}}}
\usepackage{graphicx}
\makeatletter
\def\maxwidth{\ifdim\Gin@nat@width>\linewidth\linewidth\else\Gin@nat@width\fi}
\def\maxheight{\ifdim\Gin@nat@height>\textheight\textheight\else\Gin@nat@height\fi}
\makeatother
% Scale images if necessary, so that they will not overflow the page
% margins by default, and it is still possible to overwrite the defaults
% using explicit options in \includegraphics[width, height, ...]{}
\setkeys{Gin}{width=\maxwidth,height=\maxheight,keepaspectratio}
% Set default figure placement to htbp
\makeatletter
\def\fps@figure{htbp}
\makeatother
\setlength{\emergencystretch}{3em} % prevent overfull lines
\providecommand{\tightlist}{%
  \setlength{\itemsep}{0pt}\setlength{\parskip}{0pt}}
\setcounter{secnumdepth}{-\maxdimen} % remove section numbering
\ifLuaTeX
  \usepackage{selnolig}  % disable illegal ligatures
\fi
\IfFileExists{bookmark.sty}{\usepackage{bookmark}}{\usepackage{hyperref}}
\IfFileExists{xurl.sty}{\usepackage{xurl}}{} % add URL line breaks if available
\urlstyle{same} % disable monospaced font for URLs
\hypersetup{
  pdftitle={dataVisulization},
  hidelinks,
  pdfcreator={LaTeX via pandoc}}

\title{dataVisulization}
\author{}
\date{\vspace{-2.5em}2023-02-14}

\begin{document}
\maketitle

\begin{Shaded}
\begin{Highlighting}[]
\FunctionTok{library}\NormalTok{(tidyverse)}
\end{Highlighting}
\end{Shaded}

\begin{verbatim}
## -- Attaching packages --------------------------------------- tidyverse 1.3.2 --
## v ggplot2 3.4.1     v purrr   1.0.1
## v tibble  3.1.8     v dplyr   1.1.0
## v tidyr   1.3.0     v stringr 1.5.0
## v readr   2.1.4     v forcats 1.0.0
## -- Conflicts ------------------------------------------ tidyverse_conflicts() --
## x dplyr::filter() masks stats::filter()
## x dplyr::lag()    masks stats::lag()
\end{verbatim}

\begin{Shaded}
\begin{Highlighting}[]
\NormalTok{mn\_homes }\OtherTok{\textless{}{-}} \FunctionTok{read\_csv}\NormalTok{(}\StringTok{"Data/mn\_homes.csv"}\NormalTok{)}
\end{Highlighting}
\end{Shaded}

\begin{verbatim}
## Rows: 495 Columns: 13
## -- Column specification --------------------------------------------------------
## Delimiter: ","
## chr  (2): neighborhood, community
## dbl (10): saleyear, salemonth, salesprice, area, beds, baths, stories, yearb...
## lgl  (1): fireplace
## 
## i Use `spec()` to retrieve the full column specification for this data.
## i Specify the column types or set `show_col_types = FALSE` to quiet this message.
\end{verbatim}

\begin{Shaded}
\begin{Highlighting}[]
\FunctionTok{glimpse}\NormalTok{(mn\_homes)}
\end{Highlighting}
\end{Shaded}

\begin{verbatim}
## Rows: 495
## Columns: 13
## $ saleyear      <dbl> 2012, 2014, 2005, 2010, 2010, 2013, 2011, 2007, 2013, 20~
## $ salemonth     <dbl> 6, 7, 7, 6, 2, 9, 1, 9, 10, 6, 7, 8, 5, 2, 7, 6, 10, 6, ~
## $ salesprice    <dbl> 690467.0, 235571.7, 272507.7, 277767.5, 148324.1, 242871~
## $ area          <dbl> 3937, 1440, 1835, 2016, 2004, 2822, 2882, 1979, 3140, 35~
## $ beds          <dbl> 5, 2, 2, 3, 3, 3, 4, 3, 4, 3, 3, 3, 2, 3, 3, 6, 2, 3, 2,~
## $ baths         <dbl> 4, 1, 1, 2, 1, 3, 3, 2, 2, 2, 2, 2, 2, 1, 2, 2, 2, 2, 1,~
## $ stories       <dbl> 2.5, 1.7, 1.7, 2.5, 1.0, 2.0, 1.7, 1.5, 1.5, 2.5, 1.0, 2~
## $ yearbuilt     <dbl> 1907, 1919, 1913, 1910, 1956, 1934, 1951, 1929, 1940, 19~
## $ neighborhood  <chr> "Lowry Hill", "Cooper", "Hiawatha", "King Field", "Shing~
## $ community     <chr> "Calhoun-Isles", "Longfellow", "Longfellow", "Southwest"~
## $ lotsize       <dbl> 6192, 5160, 5040, 4875, 5060, 6307, 6500, 5600, 6350, 75~
## $ numfireplaces <dbl> 0, 0, 0, 0, 0, 2, 2, 0, 1, 0, 0, 0, 0, 1, 1, 1, 0, 1, 0,~
## $ fireplace     <lgl> FALSE, FALSE, FALSE, FALSE, FALSE, TRUE, TRUE, FALSE, TR~
\end{verbatim}

\#\#First Graph We usually use ggplot function to draw plots, it has
several components as the following: 1- data: Here we specify the
dataFrame that we'll use 2- mapping: Here you specify the x-axis and
y-axis 3- Add drawing component 4- Add lables to the graph 5- Faceting
(having different graphs based on different values of a specific
variable)

\begin{Shaded}
\begin{Highlighting}[]
\FunctionTok{ggplot}\NormalTok{(}\AttributeTok{data=}\NormalTok{mn\_homes, }\AttributeTok{mapping=}\FunctionTok{aes}\NormalTok{(}\AttributeTok{x=}\NormalTok{area, }\AttributeTok{y=}\NormalTok{salesprice)) }\SpecialCharTok{+} \FunctionTok{geom\_point}\NormalTok{()}
\end{Highlighting}
\end{Shaded}

\includegraphics{DataVisualization_files/figure-latex/unnamed-chunk-3-1.pdf}
As we see above, geom\_point() represents each record in the dataframe
with a point in the plot. It usually helps to detect outliers.There are
different types of drawing. For example, we can use geom\_smooth() to
represent the relation between two variables with a curved 1ine as you
can see below:

\begin{Shaded}
\begin{Highlighting}[]
\FunctionTok{ggplot}\NormalTok{(}\AttributeTok{data=}\NormalTok{mn\_homes, }\AttributeTok{mapping=}\FunctionTok{aes}\NormalTok{(}\AttributeTok{x=}\NormalTok{area, }\AttributeTok{y=}\NormalTok{salesprice)) }\SpecialCharTok{+} \FunctionTok{geom\_smooth}\NormalTok{() }\SpecialCharTok{+} \FunctionTok{geom\_point}\NormalTok{()}
\end{Highlighting}
\end{Shaded}

\begin{verbatim}
## `geom_smooth()` using method = 'loess' and formula = 'y ~ x'
\end{verbatim}

\includegraphics{DataVisualization_files/figure-latex/unnamed-chunk-4-1.pdf}
As we see that we're able to combine two function of geom() to see the
graph

\#\#aes() aes() method also allows us to change some visual properties
in the plot including: shape, color, size, and transparency, for
example, we can change the color of the points in the plot based on
another variable (such as fireplace, beds, baths)

\begin{Shaded}
\begin{Highlighting}[]
\FunctionTok{ggplot}\NormalTok{(}\AttributeTok{data=}\NormalTok{mn\_homes, }\AttributeTok{mapping=}\FunctionTok{aes}\NormalTok{(}\AttributeTok{x=}\NormalTok{area, }\AttributeTok{y=}\NormalTok{salesprice, }\AttributeTok{color=}\NormalTok{beds, }\AttributeTok{shape=}\NormalTok{fireplace)) }\SpecialCharTok{+} \FunctionTok{geom\_point}\NormalTok{()}
\end{Highlighting}
\end{Shaded}

\includegraphics{DataVisualization_files/figure-latex/unnamed-chunk-5-1.pdf}
We can also add lables to the graph using the labs() methods as the
following:

\begin{Shaded}
\begin{Highlighting}[]
\FunctionTok{ggplot}\NormalTok{(}\AttributeTok{data=}\NormalTok{mn\_homes, }\AttributeTok{mapping=}\FunctionTok{aes}\NormalTok{(}\AttributeTok{x=}\NormalTok{area, }\AttributeTok{y=}\NormalTok{salesprice, }\AttributeTok{color=}\NormalTok{fireplace)) }\SpecialCharTok{+}\FunctionTok{geom\_point}\NormalTok{() }\SpecialCharTok{+} \FunctionTok{labs}\NormalTok{(}\AttributeTok{title=}\StringTok{"sales price VS area of homes in Minneapolis, MN"}\NormalTok{, }\AttributeTok{x=}\StringTok{"Area (square feet)"}\NormalTok{, }\AttributeTok{y=}\StringTok{"Sales Price (dollars)"}\NormalTok{)}
\end{Highlighting}
\end{Shaded}

\includegraphics{DataVisualization_files/figure-latex/unnamed-chunk-6-1.pdf}

\begin{Shaded}
\begin{Highlighting}[]
\FunctionTok{ggplot}\NormalTok{(}\AttributeTok{data=}\NormalTok{mn\_homes, }\AttributeTok{mapping=}\FunctionTok{aes}\NormalTok{(}\AttributeTok{x=}\NormalTok{area, }\AttributeTok{y=}\NormalTok{salesprice, }\AttributeTok{color=}\NormalTok{lotsize, }\AttributeTok{size=}\NormalTok{fireplace)) }\SpecialCharTok{+} \FunctionTok{geom\_point}\NormalTok{() }\SpecialCharTok{+} \FunctionTok{labs}\NormalTok{(}\AttributeTok{title=}\StringTok{"sales price VS area of homes in Minneapolis, MN"}\NormalTok{, }\AttributeTok{x=}\StringTok{"Area (square feet)"}\NormalTok{, }\AttributeTok{y=}\StringTok{"Sales Price (dollars)"}\NormalTok{)}
\end{Highlighting}
\end{Shaded}

\begin{verbatim}
## Warning: Using size for a discrete variable is not advised.
\end{verbatim}

\includegraphics{DataVisualization_files/figure-latex/unnamed-chunk-7-1.pdf}

\#\#Faceting We can also have different graphs based on different values
of a specific variable. For example, if we want make a similar graph to
what we have above but for each community (community is a variable in
our dataFrame), we can do the following:

\begin{Shaded}
\begin{Highlighting}[]
\FunctionTok{ggplot}\NormalTok{(}\AttributeTok{data=}\NormalTok{mn\_homes, }\AttributeTok{mapping=}\FunctionTok{aes}\NormalTok{(}\AttributeTok{x=}\NormalTok{area, }\AttributeTok{y=}\NormalTok{salesprice, }\AttributeTok{color=}\NormalTok{fireplace)) }\SpecialCharTok{+} \FunctionTok{geom\_point}\NormalTok{() }\SpecialCharTok{+} \FunctionTok{labs}\NormalTok{(}\AttributeTok{title=}\StringTok{"Sales price VS area of homes in Minneapolis, MN"}\NormalTok{, }\AttributeTok{x=}\StringTok{"Area (square feet)"}\NormalTok{, }\AttributeTok{y=}\StringTok{"Sales Price (dollars)"}\NormalTok{) }\SpecialCharTok{+} \FunctionTok{facet\_wrap}\NormalTok{(}\SpecialCharTok{\textasciitilde{}}\NormalTok{community, }\AttributeTok{nrow=}\DecValTok{2}\NormalTok{)}
\end{Highlighting}
\end{Shaded}

\includegraphics{DataVisualization_files/figure-latex/unnamed-chunk-8-1.pdf}

Another type of plotting is the bar plot.it's use to visualize
categorical variables

\begin{Shaded}
\begin{Highlighting}[]
\FunctionTok{ggplot}\NormalTok{(}\AttributeTok{data=}\NormalTok{mn\_homes, }\AttributeTok{mapping=}\FunctionTok{aes}\NormalTok{(}\AttributeTok{x=}\NormalTok{community, }\AttributeTok{y=}\NormalTok{..prop..,}\AttributeTok{group=}\DecValTok{1}\NormalTok{)) }\SpecialCharTok{+} \FunctionTok{geom\_bar}\NormalTok{()}
\end{Highlighting}
\end{Shaded}

\begin{verbatim}
## Warning: The dot-dot notation (`..prop..`) was deprecated in ggplot2 3.4.0.
## i Please use `after_stat(prop)` instead.
\end{verbatim}

\includegraphics{DataVisualization_files/figure-latex/unnamed-chunk-9-1.pdf}

we can also color the bars using some variables as the following

\begin{Shaded}
\begin{Highlighting}[]
\FunctionTok{ggplot}\NormalTok{(}\AttributeTok{data=}\NormalTok{mn\_homes, }\AttributeTok{mapping=}\FunctionTok{aes}\NormalTok{(}\AttributeTok{x=}\NormalTok{community, }\AttributeTok{fill=}\NormalTok{fireplace)) }\SpecialCharTok{+} \FunctionTok{geom\_bar}\NormalTok{()}
\end{Highlighting}
\end{Shaded}

\includegraphics{DataVisualization_files/figure-latex/unnamed-chunk-10-1.pdf}

sometimes, we prefer to add the mapping parameter to the geom\_chart
function and not in the ggplot function, also we can use the coord\_flip
method to swap between the horizontal and the vertical coordinates if
needed

\begin{Shaded}
\begin{Highlighting}[]
\FunctionTok{ggplot}\NormalTok{(}\AttributeTok{data=}\NormalTok{mn\_homes) }\SpecialCharTok{+} \FunctionTok{geom\_bar}\NormalTok{(}\AttributeTok{mapping=}\FunctionTok{aes}\NormalTok{(}\AttributeTok{x=}\NormalTok{community, }\AttributeTok{fill=}\NormalTok{fireplace), }\AttributeTok{position=}\StringTok{"dodge"}\NormalTok{) }\SpecialCharTok{+} \FunctionTok{coord\_flip}\NormalTok{()}
\end{Highlighting}
\end{Shaded}

\includegraphics{DataVisualization_files/figure-latex/unnamed-chunk-11-1.pdf}

As we used bar chart for categorical variables, we can use histogram
charts for continues variables as the following:

\begin{Shaded}
\begin{Highlighting}[]
\FunctionTok{ggplot}\NormalTok{(}\AttributeTok{data=}\NormalTok{mn\_homes, }\AttributeTok{mapping=}\FunctionTok{aes}\NormalTok{(}\AttributeTok{x=}\NormalTok{salesprice)) }\SpecialCharTok{+} \FunctionTok{geom\_histogram}\NormalTok{(}\AttributeTok{bins=}\DecValTok{10}\NormalTok{)}
\end{Highlighting}
\end{Shaded}

\includegraphics{DataVisualization_files/figure-latex/unnamed-chunk-12-1.pdf}

Boxplot charts are used also to represent numerical variables'
distribution across levels of a categorical variables. For example, in
the chart below, we will show the distribution of houses' prices across
different communities.

\begin{Shaded}
\begin{Highlighting}[]
\FunctionTok{ggplot}\NormalTok{(}\AttributeTok{data=}\NormalTok{mn\_homes, }\AttributeTok{mapping=}\FunctionTok{aes}\NormalTok{(}\AttributeTok{x=}\NormalTok{community, }\AttributeTok{y=}\NormalTok{salesprice)) }\SpecialCharTok{+} \FunctionTok{geom\_boxplot}\NormalTok{() }\SpecialCharTok{+} \FunctionTok{coord\_flip}\NormalTok{()}
\end{Highlighting}
\end{Shaded}

\includegraphics{DataVisualization_files/figure-latex/unnamed-chunk-13-1.pdf}

\hypertarget{r-markdown}{%
\subsection{R Markdown}\label{r-markdown}}

This is an R Markdown document. Markdown is a simple formatting syntax
for authoring HTML, PDF, and MS Word documents. For more details on
using R Markdown see \url{http://rmarkdown.rstudio.com}.

When you click the \textbf{Knit} button a document will be generated
that includes both content as well as the output of any embedded R code
chunks within the document. You can embed an R code chunk like this:

\begin{Shaded}
\begin{Highlighting}[]
\FunctionTok{summary}\NormalTok{(cars)}
\end{Highlighting}
\end{Shaded}

\begin{verbatim}
##      speed           dist       
##  Min.   : 4.0   Min.   :  2.00  
##  1st Qu.:12.0   1st Qu.: 26.00  
##  Median :15.0   Median : 36.00  
##  Mean   :15.4   Mean   : 42.98  
##  3rd Qu.:19.0   3rd Qu.: 56.00  
##  Max.   :25.0   Max.   :120.00
\end{verbatim}

\hypertarget{including-plots}{%
\subsection{Including Plots}\label{including-plots}}

You can also embed plots, for example:

\includegraphics{DataVisualization_files/figure-latex/pressure-1.pdf}

Note that the \texttt{echo\ =\ FALSE} parameter was added to the code
chunk to prevent printing of the R code that generated the plot.

\end{document}
